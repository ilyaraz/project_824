\documentclass[letterpaper,11pt]{article}

\usepackage{amsmath}
\usepackage{amssymb}
\usepackage{amsthm}
\usepackage{fullpage}
\usepackage[utf8]{inputenc}
\usepackage[pagebackref=true]{hyperref}
\usepackage{algpseudocode}
\usepackage{algorithm}
\usepackage{newclude}

\usepackage{palatino,eulervm}
\linespread{1.05}         % Palatino needs more leading (space between lines)
\usepackage[T1]{fontenc}

\newcommand{\Rbb}{\mathbb{R}}
\newcommand{\eps}{\varepsilon}
\newcommand{\set}[1]{\left\{#1\right\}}

\newtheorem{theorem}{Theorem}
\newtheorem{lemma}{Lemma}

\begin{document}
    \title{A Robust Version of Memcached}
    \author{Gautam Kamath \and Ilya Razenshteyn}
    \maketitle

    \section{Introduction}

    Memcached~\cite{memcached} is an extremely simple and useful caching system for web applications. It is implemented
    via a distributed hash table stored in the RAM of a group of servers.
    There is no communication between servers, and clients decide which servers need to be queried during a request.

    Memcached is not fault--tolerant.
    If a server fails, clients will experience cache misses. 
    Moreover, if a server dies permanently, a part of the cache is lost forever.

    While these limitations are acceptable in practice, we are interested in addressing them.

    \section{Proposal}

    We propose the following design for a distributed caching system.

    First, we use Chord~\cite{chord} to implement a DHT.
    As a result, Get/Put operations can take a superconstant time. To avoid this, the client caches
    a mapping from key ranges to the servers which hold them. As a result, if nothing fails, we have a
    constant query time. On the other hand, Chord will handle server failures, without the need for a view service.

    Second, we propose to implement fault--tolerance for the caching system. Every key--value pair is stored in
    several locations (determined using a cryptographic hash function).
    If a server fails, its data will be redistributed on other machines.

    We motivate this scheme with the following example: suppose all clients repeatedly perform the exact same query.
    The resulting load on the corresponding server greatly increases its failure likelihood.
    If the server fails, the database will receive a large amount of traffic.
    In our proposed architecture, the queries will soon be served by another machine, thus removing unnecessary strain on the database.

\bibliographystyle{abbrv}
\bibliography{biblio}

\end{document}
