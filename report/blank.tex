\documentclass[letterpaper,11pt]{article}

\usepackage{amsmath}
\usepackage{amssymb}
\usepackage{amsthm}
\usepackage{fullpage}
\usepackage[utf8]{inputenc}
\usepackage[pagebackref=true]{hyperref}
\usepackage{algpseudocode}
\usepackage{algorithm}
\usepackage{newclude}
\usepackage{titling}
\setlength{\droptitle}{-2.5cm}

\usepackage{palatino,eulervm}
\linespread{1.05}         % Palatino needs more leading (space between lines)
\usepackage[T1]{fontenc}

\newcommand{\Rbb}{\mathbb{R}}
\newcommand{\eps}{\varepsilon}
\newcommand{\set}[1]{\left\{#1\right\}}

\newtheorem{theorem}{Theorem}
\newtheorem{lemma}{Lemma}

\begin{document}
    \title{A Robust Version of Memcached}
    \author{Gautam Kamath \and Ilya Razenshteyn}
    \maketitle

    \section{Problem}
     Memcached~\cite{memcached} is an extremely simple and useful caching system for web applications.
     It is used by many high-traffic sites, including Youtube, Reddit, Zynga, Facebook, Twitter, and Wikipedia.
     It is implemented via a distributed hash table stored in the RAM of a group of servers.
     There is no communication between servers, and clients decide which servers need to be queried during a request.

     This system's simplicity comes at a price.
     If a server fails, all further queries to this server will be cache misses.
     Assuming that keys are accessed multiple times, this will cause unnecessary load on the database.
     Another issue arises if servers are assigned disjoint sets of keys.
     If there are many queries involving the same key, performance will suffer, since one server will have to handle all these queries.
     \section{Solution}
     Our solution to these problems consists of two well-known techniques in distributed systems: consistent hashing and eventual consistency.
     \subsection{Consistent Hashing}
     In our system, we distribute data among several servers.
     The obvious way to ensure a balanced allocation is to apply a hash function $h$ to the key $k_i$.
     We then assign the key to server $j = h(k_i) \bmod s$, where $s$ is the number of servers.
     The issue arises when a server joins or leaves this configuration.
     In this case, almost every key will be hashed to a new location and will have to be fetched again from the database.
     Consistent hashing~\cite{chashing} provides an elegant solution to this problem.
     In a consistent hashing scheme, servers and keys are both hashed to points on the circumference of a circle.
     A key is assigned to a server if the server is the first one reached when traversing the circle clockwise from the key's hashed location.
     As a consequence, we expect only a $\frac{1}{s}$ fraction of the keys will be relocated upon a server entering or leaving the configuration.
     This allows us to deal with server failures without introducing excessive load on the database.
     \subsection{Replication with Eventual Consistency}
     Consider a situation where all clients query for the same key.
     The throughput of the system is limited by the throughput of a single server.
     We deal with this by creating replication groups for each server site on the circle.
     A client chooses a random server in the replication group to handle its query.

     However, this introduces a new issue.
     Inevitably, different servers in the same replication group will have contradictory mappings.
     We handle this using eventual consistency~\cite{bayou}.
     In this model, servers will deal with disagreements using a procedure called anti-entropy.
     A server stores a set of diffs since the last anti-entropy session and compares these diffs with its replicas.
     If we find a set of contradictory diffs, we adopt the diff with the latest time stamp.
     While there will be a brief period of time where servers have inconsistent views of the state, these will eventually be resolved.
     This consistency guarantee is most useful in systems where reads are much more frequent than writes.
     For instance, consider a popular blog - updates to the blog are much less common than pageviews.
     Eventual consistency in this scenario would be very useful, since we are not too concerned if a viewer sees a new post slightly later than another viewer.


     \section{Implementation}
     Our implementation can be thought of as three components: client, server, and viewservice.
     We used Apache Thrift as a framework to handle RPCs between these components.
     In total, our implementation is roughly 1500 lines.

     We built three clients - a C++ client for maximum performance, and Python and Java clients as more realistic performance metrics.
     Clients support a simple put/get interface.
     Each client communicates with the viewservice to get a view and uses consistent hashing to determine which server to query.
     If we query the wrong server, we refresh our view by querying the viewservice again.

     Our server maintains a mapping of keys to values.
     Every 100ms, a server pings the viewservice to indicate it is still functional.
     Every 1000ms, a server undergoes anti-entropy with a replica.
     In our current implementation, we expect reasonable performance when the number of replicas is relatively small.
     Upon receiving a request from a client, the server first confirms that it is responsible for the relevant key.
     If not, the server's reply indicates that it was the wrong server.
     Otherwise, the server updates its local hashtable and replies with a confirmation.

     Our viewservice keeps track of which servers are currently alive, and their locations in the consistent hashing ring.
     We update the view if a new server joins the configuration or if a server dies (i.e., it stops sending pings to the viewservice).

     \section{Performance}
     We measured a few performance benchmarks to investigate the impact of these techniques.
     All benchmarks were performed with clients running a stress test, i.e., performing puts and gets on random data as fast as possible.

     Our first test involved one machine running a viewservice and five servers, and two machines each running twenty clients.
     Our system could process roughly 40000 operations per second.
     We then started another machine running five servers, and throughput increased to 60000 operations per second.

     Our next test was similar, but we had the second set of five servers be replicas of the first set.
     In this case, throughput increased to 70000 operations per second.

     \section{Challenges}
     We had issues with the hash table.

     There were also some unexpected issues involving Linux networking.
     While the C++ client was able to perform queries in rapid succession, the Python and Java clients were not.
     The Python client would hang after about 9 seconds of operation, while the Java client started raising exceptions related to networking issues.
     We discovered the issue was due to the TCP TIME\_WAIT state.
     After a socket is closed, it enters this state in order to ensure reliable delivery of packets which it does not know about yet.
     By default on Unix systems, this period is 60 seconds.
     As a result, if we open and close connections very quickly, we can reach a point where no sockets are available, due to the majority being in the TIME\_WAIT state.
     We can force these sockets to be recycled quickly by changing the system flag \verb=/proc/sys/net/ipv4/tcp_tw_recycle=.
     Since we only attempt to close after the data is sent over the connection, this should not affect correctness.
     It is unclear why this issue arises with the Java and Python clients, but not the C++ client.


\bibliographystyle{abbrv}
\bibliography{biblio}

\end{document}
